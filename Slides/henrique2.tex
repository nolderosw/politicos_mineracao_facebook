\documentclass{beamer}

% \documentclass[draft]{beamer}
% This is the file main.tex

\beamertemplateshadingbackground{red!10}{blue!10}%{green!10}
\beamertemplatetransparentcovereddynamic
\usetheme{Copenhagen} %BLOCOS!!!!!

\usepackage{beamer themesplit}
\usepackage[ansinew]{inputenc}

\usepackage{graphicx}                       % INSERIR GR{\'A}FICO
\usepackage{wrapfig}                        % INSERE O GR{\'A}FICO DO LADO DO TEXTO
\usepackage[normalem]{ulem}                 %
\usepackage{color}                          %
\usepackage{here}                           %
\usepackage{rotating}                       %
\usepackage{enumerate}                      % Este pacote {\'e} para mudar o padr{\~a}o do ENUMERATE
\usepackage{comment}
\usepackage[brazil]{babel}
\usepackage{url}


\title{Detec\c{c}\~ao de pessoas ativas e de comportamento sujeito a an\'alise em P\'aginas do Facebook de Pol\'iticos}
\author{Wesley Azevedo, Douglas Dantas}
\date{\today}
\institute{Instituto Federal de Educa\c{c}\~ao Ci\^encia e Tecnologia da Para\'{\i}ba}
\begin{document}

\frame{\titlepage}

\begin{frame}
	\frametitle{Objetivo}
	\begin{itemize}
	\item Constru\c{c}\~ao de uma aplica\c{c}\~ao capaz de minerar e julgar coment\'arios de um Post da rede social Facebook utilizando t\'ecnicas de processamento de linguagem natural (NLP) e identificar, a partir da classifica\c{c}\~ao de seu coment\'ario, pessoas sujeitas a an\'alise.
	\end{itemize}

\end{frame}

\begin{frame}
	\frametitle{ARQUITETURA DA APLICA\c{C}\~AO}
	\begin{figure}[h]
    		\centering
    		\includegraphics[height=0.6\paperheight]{arquitetura_henrique}
    		%\includegraphics[height=6cm]{figCoordEsf02}
    		\caption{Arquitetura da Aplica\c{c}\~ao}\label{arquitetura_henrique}
  	\end{figure}
\end{frame}

\begin{frame}
	\frametitle{API RESTFB}
	\begin{block}{Instala\c{c}\~ao e Documenta\c{c}\~ao}
		Para instala\c{c}\~ao da api e consulta da documenta\c{c}\~ao oficial da restfb, 
		podemos acessar o site:\href {http://restfb.com/}{http://restfb.com/}
	\end{block}
	\begin{block}{Principais Caracteristicas}
		\begin{itemize}
			\item Integra\c{c}\~ao com a vers\~ao mais recente da Graph API do facebook;
			\item M\'etodos simplificados e que "fazem o trabalho" pra voc\^e; 
			\item Funcionamento independente (n\~ao precisa/depende de outras bibliotecas java).
		\end{itemize}
	\end{block}
\end{frame}

\begin{frame}
	\frametitle{MongoDB}
	\begin{block}{Instala\c{c}\~ao e Documenta\c{c}\~ao}
		Para instala\c{c}\~ao do banco de dados e consulta da documenta\c{c}\~ao oficial da restfb, 
		podemos acessar o site:\href {https://www.mongodb.com/}{https://www.mongodb.com/}
	\end{block}
	\begin{block}{Principais Caracteristicas}
		\begin{itemize}
			\item Banco de dados n\~ao relacional (NoSQL);
			\item Orientado a documentos do tipo JSON;
			\item Sintaxe extremamente simples;
			\item Sharding (Possibilidade de compartilhamento de dados, caso encontra-se no limite de armazenemento);
			\item GridFS (Possibilidade de armazenas arquivos no pr\'oprio DB, diferente dos DB convencionais). 
		\end{itemize}
	\end{block}
\end{frame}

\begin{frame}
	\frametitle{BIBLIOTECAS UTILIZADAS}
	\begin{block}{Bibliotecas}
	\begin{itemize}
		\item PyMongo; (Acessar Banco de Dados)
		\item NLTK; (Tratamento de Linguagem Natural)
		\item String; (Manipula\c{c}\~ao do Texto)
		\item TextBlob; (An\'alise de Sentimentos)
		\item Translate; (Traduzir Strings)
		\item DateTime; (Pegar Hor\'ario)
		\item Numpy; 
		\item MatPlotLib; --------- PLOT DE GRAFOS/GRAFICOS
		\item NetworkX;
				\end{itemize}

	\end{block}
	
\end{frame}

\begin{frame}
	\frametitle{Resultados}
	\begin{block}{Resultados alcan\c{c}ados}
		\begin{itemize}
			\item Todos os POSTS/COMENT\'ARIOS da p\'agina pol\'itica.
			\item Tratamento de Strings sem as StopWords (Palavras desnecess\'arias.
			\item An\'alise de Sentimentos
			\item Incid\^encia de Usu\'arios na p\'agina pol\'tica.
			\item Determin\c{c}\~ao de Usu\'arios sujeitos a an\'alise por estudiosos.
		\end{itemize}
	\end{block}
	\begin{block}{Propostas de Melhorias}
		\begin{itemize}
			\item Plot de Grafos mais espec\'ifico.
			\item Aumentar a taxa de acertos da An\'alise de Sentimentos.
			\item Criando uma interface melhor para o usu\'ario final.
			\item Otimizando a ferramenta para deix\'a-la mais r\'apida/pr\'atica.
		\end{itemize}
	\end{block}
\end{frame}


\end{document}

